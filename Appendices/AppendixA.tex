\chap{Domain Tips} \label{apndx: domain-tips}
This appendix details the tips and further information for each behavioural domain provided to users of the Gray Matters app. The data is categorised under its' respective domain.

\section{Diet}
\textbf{Healthy food choices feed the brain and lower risk for Alzheimer's disease.}
Just like the rest of your body, your brain needs a nutritious diet to operate at its best. \textit{Focus on eating plenty of fresh fruit and vegetables, lean protein, and healthy fats.}
Eating habits that reduce inflammation and provide a steady supply of fuel are best. These food tips will keep you protected:
\newline \textbf{Follow a Mediterranean diet.}
Eating a heart-healthy Mediterranean diet rich in fish, nuts, whole grains, olive oil, and abundant fresh produce. You can treat yourself to the occasional square of dark chocolate, and if you consume alcohol, a glass of red wine.
\newline \textbf{Avoid trans fats and saturated fats.}
Reduce your consumption by avoiding full-fat dairy products, red meat, fast food, fried foods, and packaged and processed foods.
\newline \textbf{Eat a heart-healthy diet.}
What's good for the heart is also good for the brain, so by reducing your risk of heart disease, you also lower your risk of Alzheimer's disease.
\newline \textbf{Get plenty of omega-3 fats.}
Evidence suggests that omega-3 fatty acids may help prevent Alzheimer's disease and dementia. Food sources include cold-water fish such as salmon, tuna, trout, mackerel, and sardines. You can also supplement with fish oil.
\newline \textbf{Eat across the rainbow.}
Emphasize fruits and vegetables across the color spectrum to maximize protective antioxidants and vitamins. Daily servings of berries and green leafy vegetables should be part of your brain-protective regimen.

\section{Physical}
\textbf{Research shows that people who exercise regularly have a lower risk for multiple degenerative diseases, including Alzheimer's.}

\textit{Quality and quantity} of the physical activity are important. 

Incorporating physical activity into your daily routine is linked to:
\begin{enumerate}
	\item Lower stress levels
	\item Better physical health
	\item Improved sleep
	\item Enhanced brain performance
\end{enumerate}

\section{Sleep}
\textbf{Good sleep is linked to better mental performance during the day, and lower risk for Alzheimer's disease.}

Promote better sleep by:  
\begin{enumerate}
	\item \textit{Avoiding caffeine} within 4 hours of bedtime 
	\item Finishing exercise \textit{3 hours or more} before bedtime 
	\item Creating a \textit{quiet and dark} environment
	\item Maintaining a \textit{regular bedtime} and wake-up time each day, to help set your internal 'clock' for better sleep.
\end{enumerate}

\section{Social}
\textbf{Research shows that people who have a socially engaged lifestyle, have \textit{higher scores on cognitive tests}, and are at a \textit{lower risk for dementia} and Alzheimer's disease.}

\textit{Quality is more important than quantity}

Feeling emotionally supported and having low conflict with others is linked to:
\begin{itemize}
  \item Lower stress
  \item Better physical health
  \item Better brain performance
\end{itemize}

\section{Cognitive}
\textbf{Stimulating your brain by mentally challenging yourself builds \textit{'cognitive reserve'}, associated with lower Alzheimer's risk.}

\textbf{Novel mental exercises} are very important (e.g. memorizing a recipe or grocery list, learning new words or a foreign language, doing arithmetic problems, helping kids/grandkids with homework). 

Also important are \textbf{cognitively stimulating activities}, things like volunteering, joining a book club, playing a musical instrument, attending a lecture or concert, or debating friends on the hot topic of the day! 

Keep it fun so you'll stick with it!"

\section{Stress}
\textbf{Psychological stress increases risk for Alzheimer's disease.}

Lower your AD risk by taking a moment to \textit{manage your daily stressors} and building into each day a time to reflect and be mindful of what is truly most important in life. 

Try doing this by \textit{increasing awareness }of your mind and body and allowing yourself to take a break from the stress. 

\textit{Taking a walk, talking to a friend, or contacting a loved one} can provide the opportunity to be both aware and a time-out to reset your stress.

\section{Extra: Drinks}
\texttt{This section was not included in the original RCT study version of the Gray Matters app, however, was developed for the second iteration released to the public.}

\textbf{The human brain is composed of 95\% water, blood is 82\% water, and the lungs are nearly 90\% water}

\textbf{How important is water to us?}
\newline A 2\% drop in body water can cause a \textit{small but critical shrinkage of the brain}, which can impair neuromuscular coordination, decrease concentration, and slow thinking. 

Dehydration can also reduce endurance, decrease strength, cause cramping, and slow muscular response.

It is suggested that the average person requires a \textbf{\textit{minimum of 8-to-12 glasses of water per day}}. If you are doing exercise, in a hot climate, or consuming alcohol, make sure to add a few extra glasses!
