\chap{Introduction} \label{chapter: introduction}

\section{Public Health}
Public health refers to all organised measures, both public and private, to prevent disease, promote health, and prolong life among the population as a whole \cite{WorldHealthOrganisation2016}.
The focus of public health is not only disease prevention, it is also concerned with the promotion of physical and mental wellbeing. The 3 main functions of public health, as stated by the \citeauthor{WorldHealthOrganisation2016} , are:

\begin{itemize}
	\item Health assessment and monitoring of populations at risk to identify health problems and priorities.
	\item Creation of public policies designed to solve the identified problems and priorities.
	\item Ensuring that the everyone has access to the care, in a cost-effective manner, which includes health promotion and disease prevention services \cite{WorldHealthOrganisation2016}.
\end{itemize}

Achievement of these functions requires the implementation of focused, and wide-scale solutions. In the traditional medical model, a clinician treats a disease, one patient at a time, whereas, public health aims to prevent the disease from ever occurring. It is working. The population as a whole is healthier, has a higher quality of life, and is living longer \cite{Uhlenberg2009}. However, the last benefit presents a new challenge.

\subsection{An Ageing Population}
As the average age of the population increases, so does the prevalence of chronic age-related conditions. The incidence of physical ailments, such as arthritis, osteoporosis and cataracts increases. Whilst undesirable, the onset and progression can be monitored closely and many treatments exist. More insidious, however, are the conditions which exist in the neurological domain, such as Parkinson's disease and dementias. Without presenting with obvious physical symptoms, these neurodegenerative conditions slowly creep into a person's life, before ultimately rendering them totally dependent on others to survive. This loss of personal independence, personality, and ultimately the person, warrants fear and action.

\section{Dementia}
Dementia is an umbrella term for a wide range of cognitive symptoms which ultimately result in the progressive deterioration of cognitive functioning, beyond the level expected in normal ageing. The onset of dementia can be attributed to a number of conditions and diseases, with the most common cause being Alzheimer's Disease (AD), estimated to cause between 60\% to 80\% of all cases \cite{2015AlzheimersDiseaseFactsFigures}.  

\subsection{Symptoms}
The cause of dementia does not change the symptoms, but may alter the order of presentation during the progression of the disease. Common symptoms include memory loss, difficulty communicating, difficulty performing complex tasks (including planning and organising), problems with orientation, personality changes, inability to reason, agitation, paranoia, and in later stages of AD hallucinations are common \cite{NationalHealthService2015}.

\subsubsection{Management}
As dementias are currently incurable, treatment options are currently limited to symptom management. Pharmaceutical treatment involves drugs which slow the progression of the disease, and tranquilliser based drugs which reduce stress, anxiety, and aid sleep. As the cognitive and physical abilities of Persons with Dementia (PwD) begin to falter, they become increasingly dependent on others. 
Carers play a large role in the lives of PwD, helping to remind and assist with various activities of daily living (ADL), such as eating, dressing and taking medications. Due to the shift in age demographics, the world is moving towards a situation where there are not enough young to care for the old. In the effort to find solutions, technology has stepped forward with a number of novel proposals.

\section{Technology's Role in the Health Paradigm Shift}
Technology is in a position to alleviate the burden from existing and future carers, by aiding or automating the basic non-physical functions of carers, freeing them to focus on human-centric tasks.
Technology solutions such as remote vitals-monitoring, medication dispensers, cognitive aids and context-aware smart-homes, all provide distinct advantages over traditional human-contact centric care models. They also have distinct disadvantages, such as cost, scalability, and adoptability. The contextually-aware smart homes are at the bleeding edge of technology research in dementia care, but are currently experimental and would require significant refinement before they are suitable as a public health solution, to be scaled up for the masses.
For a more immediate effect, focus has been targeted to an existing and everyday technology; the \textit{Smartphone}.

\section{Smartphone}
In the pockets of over 3 billion people across the globe \cite{EricssonMobilityReport2016}, and capable of running bespoke software, the smartphone is the perfect everyday technology from which to build a public health solution. 

\subsection{Monitoring}
Smartphones are always connected devices, with a range of sensors which may facilitate remote monitoring of any perceivable activity. For PwD, this could be the monitoring of sleep quality, detection of wandering episodes, or a the detection of potential harmful event such as a fall.

\subsection{Cognitive Aids}
Marked loss of cognitive functioning and memory are undoubtably prominent features of dementia. Fortunately, cognitive aids do exist to help the planning, scheduling and reminding of tasks. Technology research has expanded the vision from simple time-based alarms to context-aware smart homes that aim to issue and guide users through various tasks in the home, through activity recognition.

\subsection{Modify behaviours}
Not only can technology aid in the treatment effort, new and emerging work suggests that technology may play an integral role in the prevention of the disease, through the study and modification of risk factors. Current research estimates that genetic risk factors of Alzheimer's disease account for only one third of the risk, and that the majority of risk is due to lifestyle factors \cite{Ridge2013}. 

\subsubsection{Modifiable Factors}
Fortunately, unlike genetics and various environmental factors, lifestyle factors can be modified, through changes in behaviour. These changes involve positive alterations to diet, exercise, sleep, cognitive exercise, socialising and stress reduction. A range of smartphone apps cater to these areas individually [ref], however, none have targeted all areas at once for the specific purpose of reducing future disease risk.
%TODO: Add REF
\subsection{Adoption}
Whilst the mantra \textit{`build it and they will come'} may be true for for some, the adoption of new and useful technology is not guaranteed [ref]. This is especially apparent in the elderly and the cognitively impaired. Whether it be for treatment or prevention, it is important to understand and encourage adoption, and learn how it may be maintained over time.
%TODO: Add REF

\section{Rationale and Aim of this research}
Given the potential impact of smartphones in the treatment and prevention of Alzheimer's Disease, this Thesis seeks to answer the following questions:
\begin{itemize}
	\item Can a smartphone improve the quality of life for dementia sufferers, and how?
	\item Can smartphone sensor-technologies bridge the gap between experimental context-aware systems and deployable solutions for PwD?  
	\item Can an app positively influence behaviours of the end-user?
	\item Can smartphones be used to mitigate the risk of diseases?
\end{itemize}

\section{Thesis Overview}
% Chapter 1
Chapter \ref{chapter: introduction}, this chapter,  details the major motivating factors for the research and presents a brief background in each area. In addition, the chapter details a structural overview of the thesis and the author's current contributions. 

% Chapter 2 - Literature Review
Chapter \ref{chapter: lit-review} discusses the opportunities for mobile computing in the current dementia epidemic, exploring the potential impacting roles in both the treatment and prevention of the condition. The chapter is structured as an overview of dementia, its numerous causes and resulting symptoms. The chapter then explores technology's current role in the management of the disease, diverging into 2 distinct areas: treatment and prevention. A literature review of each area is presented, with each concluding with the author's identified areas for contribution.

% Chapter 3 - TAUT treatment
Chapter \ref{chapter: treatment-framework} explores the area of context-aware mobile computing, which is applied to the area of reminder solutions for PwD. The development of a mobile sensor based context-recognition system, aims to improve reminder adherence, through the recognition of contexts which are suitable and unsuitable for the delivery of reminders. The app and sensor framework is tested and evaluated by a group of healthy adults, highlighting areas for improvement. The refined app is then deployed to 30 persons with dementia as an intervention tool for 12 months. Post-study analysis of the sensor and usage data is performed, culminating in the testing and evaluation of various classification models for the system. 
Related publications include: \cite{Hartin2014-EMBC, Hartin2014-WAGER, Patterson2015, Cleland2014-IWAAL, Cleland2015-mHealth, Nugent2014-aaic, Behrens2015}.

% Chapter 4 - Prevention Framework
Chapter \ref{chapter: prevention-framework} details the design and development of a smartphone facilitated behaviour change framework. The Chapter begins by exploring the current literature on the science of behaviour change, and identifies areas in which the process can be aided though technology injection. A framework and process guide is then created from the literature. The resulting framework is then applied to the area of Alzheimer's Disease prevention, producing an app for the iOS and Android platform, which is subsequently evaluated in the subsequent chapters. 
Related publications include: \cite{Hartin2015-JMIR, Hartin2014-IWAAL, Hartin2015-ICOST, Hartin2014-AAIC, Hartin2015-AAIC, Hartin2015-mHealth, Norton2015-TRCI, Norton2015-AAIC, Weyerman2015}.

% Chapter 5 - Evaluation Through Expert Review
Chapter \ref{chapter: prevention-evaluation} continues the work from the preceding chapter, and details the  critical evaluation of the developed app, and focuses heavily on the objectivity of the process. In addition to the expert review, content analysis is performed on a large number of available apps in the mHealth domain. These evaluations result in the exposure of common attributes amongst successful apps, from which a number of recommendations are made by the author for future works. 
Related publications include: \cite{Hartin2015-JMIR, Hartin2014-IWAAL, Hartin2015-ICOST, Hartin2014-AAIC, Hartin2015-AAIC, Hartin2015-mHealth, Norton2015-TRCI, Norton2015-AAIC, Weyerman2015}.

% Chapter 6 - Results from RCT
Chapter \ref{chapter: prevention-rctresults} is the culmination of the preceding chapters' efforts, and details the clinical and behavioural effects resulting from use of the app in a randomised control trial with over 146 persons, over a duration of 6 months. 
Related publications include: \cite{Hartin2015-JMIR, Hartin2014-IWAAL, Hartin2015-ICOST, Hartin2014-AAIC, Hartin2015-AAIC, Hartin2015-mHealth, Norton2015-TRCI, Norton2015-AAIC, Weyerman2015}.

% Chapter 7 - Conclusion and Future Work
Chapter \ref{chapter: conclusion} summaries the results of the studies in chapters \ref{chapter: treatment-framework}, \ref{chapter: prevention-framework}, \ref{chapter: prevention-evaluation}, and \ref{chapter: prevention-rctresults} from which a number of recommendations are aimed at behavioural researchers, health practitioners and research funding bodies. The thesis concludes with the author's identified areas for future work.