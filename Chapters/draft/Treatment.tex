\chap{The Role of Technology in Dementia Treatment}
This chapter will explore the role of technology in the treatment of dementias. Firstly, challenges of living with the disease are explored, highlighting key targets for technology to aid the burden. 

\section{Introduction}
This chapter will explore the challenges faced in the efforts to treat and prevent dementias. It will then explore the opportunities for technology in these areas. 

\section{Prevention}
Need: Promoting self-management of health, healthcare education, promoting behavioural change.
Neurological processes of dementia
Highlight Dementia Prevention efforts
Risk factors

\subsection{Risk Factors}
\subsubsection{Genetic Risk Factors}
i. Overview of genetic risk factors
\subsubsection{Environmental Risk Factors}
ii. Overview of environmental risk factors
\subsubsection{Lifestyle Risk Factors}
ii. Overview of environmental risk factors
\subsubsection{Modifiable Risk Factors}
iii. Examination of modifiable risk factors
Modifying risk via behaviour change

\subsection{Behaviour Change}
To modify risk, the risky behaviour must be changed. To do this we need to change behaviours. Large area in physchology on the subject. Many models describe these processes. 
\subsubsection{Models}
Overview of main models of behaviour change i.e. TTM and change wheel
Identify key components
i.e Education, Self-realisation, motivations, rewards etc.

\section{Technology as an enabler}
Supporting role of technology as a method to deliver or monitor participation. 

\subsubsection{Education}
\subsubsection{Monitoring}
\subsubsection{Engagement}
\subsubsection{Reward}

\subsection{Existing technology approaches}


\subsection{Health Education Interventions}
What they are?
How they work?
What are their aims?
How are they assessed?

\subsection{Behaviour monitoring applications}
Wearable activity monitors
Fitbit, Strava, Withings etc etc.
Hardware - Wearables

\subsection{Challenges}
Initiating interest
Adoption of technology
Maintaining engagement
Quality of information delivered to user
Validity of self-reporting

\subsection{Opportunities}
Mass adoption - Trending, remove stigma
Personalisation - Truly personal feedback based on reported behaviours
Forecasting - Forecast health 
Improve engagement - Tie into model
Improve research methods
