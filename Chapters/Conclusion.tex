\chap{Conclusion and Future Work}
\label{chapter: conclusion}

This chapter concludes and summarises the results from thesis studies, summarising the findings and producing a number of messages for various research fields and practices.

\section{Thesis Summary}
Through this body of work the underlying theme has been to exploit the ubiquitous nature of smartphones to improve in health and wellbeing outcomes. This has been demonstrated in two distinct cases studies which span both the treatment and prevention areas of Alzheimer's disease. 

Chapter \ref{chapter: treatment-framework} focused on using the ever available smartphone as the platform for an assistive reminder tool for persons with dementia. The adoption of the solution from the cohort was encouraging, yet over-time exposed fundamental issues regarding technology adoption with the elderly and cognitively impaired. Despite a lack of widespread adoption at the end of the 12 months, the data from the study permitted the concept of using smartphone sensors to predict if a reminder will be acknowledged or missed, to be proven. The following question, yet to be answered, is if this will improve the acknowledgement rates of reminders, and if that will consequently result in improved adherence to the actual tasks being reminded. As such, the models described in the chapter need be implemented and evaluated with another, comparable, cohort.

Chapters \ref{chapter: prevention-framework}, \ref{chapter: prevention-evaluation}, \ref{chapter: prevention-rctresults} were focused heavily in the areas of dementia prevention through behaviour change, facilitated by a smartphone app. The works demonstrated how disease risk could be mitigated by applying a process framework developed by the author, to a disease area. The result of which produces an app which guides and encourages users, step by step, to change disease associated behaviours, resulting in continual health improvement. Unlike the work in chapter \ref{chapter: treatment-framework}, the study participants had no such percieved barriers to adoption, and as such the case study truly exploited the pervasive nature of the smartphone, yielding every advantage offered to integrate with an individual’s daily life and routines. 
 
\subsection{Contributions to Knowledge}
\begin{itemize}
	\item Demonstrated use of sensor data to infer opportune moments to deliver reminders, for PwD and healthy individuals.
	\item Developed mobile-app behavioural change framework and guidelines.
	\item Detailed impartial approach to critically evaluate and rate mHealth apps.
	\item Demonstrated behavioural effects of applied framework.
	\item Demonstrated clinical efficacy of applied framework.
	\item Identified areas for future mHealth-supported behavioural research.
	\item Identified areas for future research of reminder tools for PwD.
\end{itemize}

\subsection{Future Work}
Throughout the works, during reviews of existing literature, analysis of study data, and interviews with study participants, a number of areas were identified by the author in which future study could improve, or elicit new and interesting, findings.

\subsubsection{Advanced Personalisation and Feedback}
Interviews and questionnaires performed by the participants in the Gray Matters study indicated that \textit{truly}personalised feedback was greatly desired. Specifically, participants wanted to see personalised progress reports, based upon their own efforts, rather than those imposed by global recommendations. Whilst the study collaborators had foreseen this, it was agreed that it would detract from assessing motivation for established goals. In retrospect however, it may have been possible to facilitate both, as discussed in Section \ref{subsection: gm-future-personalisation}. As the demand for such a feature was so high, it is anticipated that personalised goal setting may significantly increase both adoption and user retention, two key factors for successful behaviour change.

\subsubsection{Artificial Intelligence Coaching}
In the Gray Matters study, participants had reported that there were various times when a fact and suggestion pair was not fully understood, or had raised additional personal queries, for which the app had no answers. Often the participants would turn try to reach out to the study investigators, in some cases eventually turning to online search websites for answers. The queries were simple, and typically related to diet and exercise. An example scenario:
Daily suggestion presents: 'Eat 2 cups of brown rice instead of chips with dinner tonight'. User would like to know if white rice is Ok, as they do not own brown rice. 
It is a simple query, and the answer is relatively inconsequential, however, a lack of answers extrapolated over numerous users and time, will result in loss of users, and thus loss of impact. A human operated query answering facility to service a large user based is also not feasible. However, great progress is currently being made in natural language engineering, enabling the ability to answer open questions through databases of expert knowledge \cite{HIRSCHMAN2001, Fader2014}. Application of these techniques, whilst advanced, will allow researchers to develop a deeper suggestion base, and provide a much higher level of support to end-users.
 
\subsubsection{Evaluation of sensorised reminder delivery system}
The work performed in Chapter \ref{chapter: treatment-framework} needs to be extended to fully achieve the original goal of improving adherence to reminders for PwD. The current work provides a solid technological base from which the models may be implemented on a smartphone, or cloud. The app was written in a modular approach, with the intension to be extended by the author and others in the field. Currently, the base application has been extended by \citeauthor{Patterson2015} who aims to use the platform to assess and eventually improve task completion followthrough, once a reminder has been issued \cite{Patterson2015}.

\subsubsection{Carer-focused Apps}
Carers of PwD are often family members, whose age would be similar as the middle-aged cohort in the Gray Matters study, who posed very minimal barriers to successful technology adoption. Given the opportunity to expand upon the work performed in Chapter \ref{chapter: treatment-framework} the author would desire to create a carer orientated solution, removing the need for the PwD to have any interaction with technology.

\section{Messages for Behavioural Scientists}
\textbf{Harness the Smartphone.}
Changing behaviours may be the key to tackling the leading causes of mortality and morbidity, such as poor diet choices, lack of exercise, smoking and alcohol consumption. Diet and exercise apps to quantify efforts do exist, some with their foundations in exercise and nutritional science, yet there is a gaping void where behavioural and psychological science based apps should be. 
The smartphone, like no other tool before it, has become part of everyday life, embedded into the behaviours of millions, giving direct access to quantifiable data. Through the understanding of behaviours on a global scale, the development of all health interventions will improve. It is of the authors opinion, that the merging of behavioural science with ubiquitous computing, currently in the form of smartphones, stands to empower the individual, and revolutionise healthcare on a global scale. 

\section{Message for Technologists}
\textbf{Technology is the question, not the answer.}
Whilst technology may hold some of the answers, it is not \textit{the} answer. Before applying technology to a problem area, ask, \textit{does it really belong there?}
A great level of diligent research must be performed, drawing knowledge from every available perspective, through collaboration and interdisciplinary research. Whilst it is true that technology-led health-orientated projects often advance the state-of-the-art, it seems seldom that they actually prove efficacy over more traditional, non-technology based approaches. With increased focus on efficacy and impact, funding and reputations are at stake. It is up to us, the technologists, to prove that the field deserves the spotlight that it gets, through true collaboration, not discordant co-operation.